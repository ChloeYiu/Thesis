\chapter{Introduction}

\section{Motivation}
More demand for automatic assessment.

To maintain storing validity, need to ensure system is unbiased.

\section{Approach}

Previous research has proved concept activation vector (CAV) \nomenclature[Z]{CAV}{Concept Activation Vector}
is a good approach for feature-based model, extracted through a simple linear classifier with no weighting.

There are however, different types of graders, and different ways to extract the CAV.

Investigate how the gradient distance pattern differs across models, and how the use of different CAV extraction method affects the CAV performance.

Additionally tries to narrow down factors affecting gradient distance pattern for a deeper understanding.

\section{Report Outline}
This report consists of 7 chapters with the following structure:
\begin{itemize}
    \item Chapter \ref{chap:assessment} introduces the role of an unbiased system in a spoken language assessment system, and highlighted the current development in the field.
    \item Chapter \ref{chap:graders} describes the different types of graders which would have its fairness being measured.
    \item Chapter \ref{chap:cav} outlines the steps of CAV extraction.
    \item Chapter \ref{chap:setup} describes the construction of training, calibration and testing data, alongside explanation on model biasing, factor isolation setup, and justificatin of performance metrics and hyper-parameters.
    \item Chapter \ref{chap:results} presents the results of the experiment, and discusses the implications.
    \item Chapter \ref{chap:conclusions} concludes the report and discusses future work.
\end{itemize}